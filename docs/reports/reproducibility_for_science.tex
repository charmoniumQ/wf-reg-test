

Science owes its self-correcting nature to certain norms of research in the scientific community, including organized skepticism and communalism~\cite{merton_sociology_1974}.
Satisfying these norms in a computational context requires reproducible experiments.

%More than 90% of scientists surveyed across all fields use research software and 50% develop software for their research experiments~\cite{hettrick_softwaresavedsoftware_in_research_survey_2014_2018}.

\textbf{Organized skepticism} means scientific claims should be exposed to critical scrutiny before being accepted~\cite{merton_sociology_1974}.
One form of critical scrutiny is reproducing the experiments supporting a claim.
When experiments in modern science are not reproducible, they lose important validation opportunities.
Unlike physical experiments, computational experiments have no requirement for a steady hand, calibrated sensors, or uncontaminated dishware; one would expect they should be easier to reproduce.
Nevertheless, computational experiments are often a source of irreproducibility, for example~\cite{mytkowicz_producing_2009,neupane_characterization_2019}.

\textbf{Communalism} means that any scientist can freely modify and reuse methods of research developed by someone else.
For computational research, communalism requires that the computational methods be open source\footnote{There are many definitions of ``open source''; here, we mean the source code must be available for free for research purposes.} and reproducible by other researchers.
Without this, scientists have to duplicate each other's work, which creates another avenue for errors and mistakes.

When either of these norms are lacking, computational methods can yield wrong results in basic research, for example~\cite{herndon_does_2014,miller_scientists_2006,wang_retracted_2012,neupane_characterization_2019,qiu_retraction_2019,villanueva_no_2021,mytkowicz_producing_2009}, further eroding public trust in science.

Outside of basic research, reproducibility of experiments is important in applied science.
Engineers use computational experiments to simulate the behavior of a physical part.
Simulations are rapidly improving, so they may want to rerun a simulation done in the past with newer techniques or with different parameters.
The physical part may have a lifetime measured in decades, but the software simulation is much more fragile, lasting only years.
If the computation is not reproducible, they cannot easily rerun the simulation; they must either attempt time-consuming digital archaeology or rewrite the simulation from scratch.
Irreproducibility hinders day-to-day operations in applied science.
